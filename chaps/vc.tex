\chapter{Vehicular Cycling}

\section{Overview}

Vehicular Cycling is a bicycling philosophy which purports to be the "safest" way to ride.

This is based upon the results of a two part study done by Ken Cross in the 1970's.

Its mantra is that "Cyclists fare best when they act like and are treated like
drivers of vehicles."

Though their catch phrase sounds slightly stupid, we'll see that in practice,
calling them stupid is an act of kindness. In fact, VC followers have done more
damage to cycling than any other group. 

This is because of their selfish desire to mix with traffic which has denied
thousands if not millions the benefits of cycling. It has also resulted in
making road conditions much more hazardous for cyclists, and it has fanned the
hatred of motorists for cyclists.

The most embarassing part of vehicluar cycling is that most VC riders don't
ride often. In fact, their founder, Mr. Forrester has been quoted as saying
that bicycles are inferior to motorised vehicles.

VC followers have been known to bully anyone who disagrees with them with name
calling and worse. They pose as experts in trials and to governments even
though they have no special credentials and they have done no novel research on
cycling. They pretend to be the sole custodians of cyclig knowledge even though
most of their "knowledge" is opinion and much of it flies in the face of common
sense, logic, everyday experience, and even the sole study that they base their
riding style on.

\section{Normal Cycling Advocates Agenda}

I call Cycling Advocates "Normal" when they are honest about their agenda, and they stand for something that makes sense to most people with out a lot of special training aka indoctrination.

Normal Advocates main belief is that the more people who bicycle, the better things will get for all cyclists. For those of us who grow up in a representative republic like the United States, this is pretty obvious.

Of course, VC riders disagree.

CITATION.

The main way that us, normal cycling advocates wish to attain more ridership is
the same means that has accomplished everything else which is more spending on
bicycling especially government spending. There is a tight coorelation between
the amount of government spending on the mode of transportation and the number
of people who do it. Again, for most this is totally obvious.

VC riders, of course, disagree.

CITATION.

With our cards on the table, we can turn towards the VC agenda.

\section{The VC Agenda}

There are two agenda for VC advocates, their public agenda and their hidden one. 

Why two agendas?

Because they know that most people will not be in favor of their hidden agendas.

VC advocates claim to be in favor of safety and rights to the road.

What they actually want, however, is to ride as quickly as possible IN TRAFFIC, ONLY. This is only on the weekends, because the other dirty secret of VC advocates is that they are actually motorists in lycra. Since they only ride a short time, if at all, each week, they always favor motorists over cyclists.

What? Why the hell would someone want to play in traffic. Don't we know that it's dangerous?

VC advocates claim that their unique style of riding, which can only learned by
taking a class, guarantees the safest riding style.

This totally untrue for reasons that we will examine in detail below.

For now, it's sufficient to realize that VC riders claim to be safer because
they know that Americans are obsessed with safety, and that if they could just
convince enough people that they have the holy grail of safety, nobody would
disagree with them.

When shown even more safer styles of riding, they'll drop safety like a hot
potato and throw out other excuses. They quick dissasciation from safety shows
that they weren't after safety at all, but rather speed and their sick (and
dangerous) fetish of mixing with traffic.

They will also say that "safety has to be balanced with convenience". True, but
once we start risking our lives and other people's lives, which are very real,
for this fuzzy notion of convenience, we start to have much different debate.
How many lives are we willing to risk for the fuzzy notion of convenience?

Plus, what does convenience even mean?

From a VC point of view, it means to ride as quickly in high speed traffic. By
in traffic, I don't mean next to traffic, but by "taking a lane" on roads where
the average speeds are fifty or more. Since the faster cyclists do about twenty
miles an hour, the speed differencial is thirty miles an hour or more. 

The best part about VC arguments is that they end with amnesia to the VC
opponant. Once their riding style is shown as unsafe in on forum, they will
move on to another and start talking about they ride the way that they do
because it is the safest style possible. Ever. 

\section{VC Methods Overview}

VC people are like everything else, when you don't know about them you don't
see them, but once you learn who they are, they're everywhere. 

Basically anywhere they were going to make any progress with government spending and building of facilities for cyclists, they come en mass and attack it. They have a myriad of arguments that will be summarized later. 

They are also fond of suing cities.

Some of them even got top positions in government bicycle planning positions
such as CITATION from Dallas.

In many bicycle clubs and coalitions they will attack any kind of progress often with subtle and vieled, but always annoying, attacks.

\section{Ad Hominem}

Everyone likes a good ad hominem attack. Most of us stopped calling people names when were in grade school. 

Not VC cyclists.

They use name calling as a short cut to actual debate.

\subsection{Fear}

On of the favorite arguments they like to use is fear. They like to pretend that everyone who wants a quieter and more comfortable place to ride is phobic.

This is partially true. In fact, I think that fear is a good thing. To quote
Buffy Summers, "my emotions are total assets."

In fact, normal people see those who are NOT afraid of getting rear-ended as lunatics.

But also, I might add, what if we are fearful? Why is this a bad thing? Most cyclists who die in the US do so at the hands (wheels?) of a motorist.

Also, VC, at least before they are against it, are all about making things
safer. We went to the moon, surely the same, can do, American engineering that
got us there can make things safer here, and that's good, right?

I see fear as a natural kind of protection, and I feel that it's foolish to get
rid of that fear. As we'll see later the fear of being rear-ended has some
basis in reality. Getting rid of this fear will put us into danger.

The irony is, though, that those who want facilities, normal advocates, ride much more often than VC riders. Normal advocates ride in places that VC riders feel is too difficult or dangerous to ride. When pressed on this VC riders get angry:

"I was right with you until you suggested the ride up Texas (a huge, noisy hill). You have pissed me off twice today."

Note that they didn't even come up with a reasonable response? Funny.

VC cyclists will argue that instead of taking dangerous routes, they will drive. This is fine, but if you drive, then you are a motorist.

\subsection{Superstious} 

Often, we are told that we are full of superstition. Again, this comes up a lot in our fear of rear-ending, but in other things, too.

My question is that if we are superstitious, surely VC is not. IF so, VC should rise to their own standard. Someone who is not superstitious is scientific.

Which is strange because unlike VC people, scientists rarely speak in terms of absolutes. Even when presenting data, they are quick to point out that their conclusions are preliminary and are subject to further investigation.

On the other hand, VC is written in stone. To quote,

"VC hasn't changed in thirty years because the vehicle code and the patterns of traffic have not changed."

This is so funny because facilities he rails against have been studied for safety an they have gotten much better. Note that as a scientist, he completely ignores these findings and continues to speak as if it's 1974.

\subsection{Anti-motorist}

This is just confusing. Why does this matter? 

I mean, if an advocate is FOR one thing and his opponants are against him, then it makes sense that he's anti-them.

The funny thing is that most cyclists also drive. That disqualifies most of us unless we hate ourselves (which I definitely don't. Just ask anyone who knows me).

Also, most motorists kind of feel sorry for cyclists and want to build bike facilities to make us feel better. So in a way motorists and cyclists work together.

Also, the motorists who do hate cyclists still wants to get us off the road, so they, too are in favor of facitilties.

So you have a big happy family of cyclists and motorists all working for a common good for us all.

Enter VC advocates. They are the ones who whine about how "motorists are taking our rights to the road." 

Yet somehow this is not anti-motorist.

Then they talk about how anti-motorist the "unthinking" cycling advocates are.

Um, no. We were all doing well until the VC people came in and confused the motorist by telling them that, "no, keep your money, we don't need your stinkin' facilities."

The thing that grates the most is the fact that while posing as a cycling advocate, the VC Quisling drove a car to the meeting to argue with all of us who rode their bicycle there!

\subsection{Unthinking} 

Now this is just plain lazy. If Forrester is too lazy to come up with any other kind of argument, he'll call someone an "unthinking cyclist." This attack is usually inversely proportional to how brilliant the attack is. That is the more brilliant the attack

% Sat Oct 22 17:44:52 PDT 2011
