\chapter{Riding Tips}

\section{What Riding Style is Best}

There are as many different riding styles as there are number of riders.

The important thing about riding style is to pick one that make you happy and comfortable.

Don't allow anyone to tell you how to ride!

Watch out for a self-professed experts who claim to have the magic key that will make you invulnearable on a bike.

Rather you should use your own reason and instincts to ride. At the same time,
you should read widely, watch other cyclists, and discuss riding strategies
with others. Overtime, with experience, you'll feel more comfortable on a
bicycle, and you'll be happier riding.

I feel like I learn something new from each cyclist that I encounter. 

\section{Salmoning}

Salmoning refers to riding against the flow of traffic. It's most commonly done
by those who ride BMX bicycles, but it is an option to all kinds of cyclists.

The main advantage of salmoning is the added feeling of safety due to the
ability of the "fish" to see the traffic as it comes and to get out of the way
if someone tries to ride you off the road.

Another added impetus toward salmoning is antiquated lessons from the
nineteen seventies which taught people to ride toward traffic.  

There are many who say that salmoning is dangerous, but I have not seen a study on this to date.

One of the main problems with salmoning, however, is that the "fish" will come
into conflict with cyclists of every other riding style. 

Also, since the salmon can see oncoming traffic, while other cyclists can not.
Thus, it's generally agreed upon that the salmon is mostly responsible for
safety and must ride further out into traffic. This double danger often negates
the percieved safety of salmoning. 

\section{Sidewalk Riding}

This is another riding style that is often derided, and seen as being highly dangerous.

In my opinion, riding on the sidewalk in either direction need not be dangerous.

The main things that causes conflicts in this riding style is intersections.

But these need not to be fatal if one stops and looks both ways before
crossing. I suggest going cautiously into intersections. In many intersections,
it's best to walk your bicycle. Also, watch out for driveways. I slow down at
the edge of each one and look both ways before going forward.

Also, watch out for pedestrians. Sidewalks are for  pedestrians, and they have
the right of way. Cycling is about kindness and low impact so I always walk my
bicycle when passing someone--unless the person looks extra douchy. 

Of course, this isn't practical for going fast or for long distances, but
sidewalk riding is not about that. It's a way for cyclists to relax and to get
comfortable on their bikes.  

I usually ride up really scary hills on the sidewalks. I notice that it allows
me to ride slower, and to relax more. The slow speed of the climb gives me time
to watch out at intersections.

\section{Road Riding}

In some mystical places like Copenhagen, road riding is seldom necessary. 

However, in most of the United States things are different.

Love it or hate it, for the time being, we have to suffer from lack of public
support (aka money) for infrastructure. Thus, many of us have to spend at least
some of the time on smelly and noisy roads.

Despite this, I still enjoy everyday of cycling to the hilt. 

Road riding is best learned by doing. I suggest taking the quieter roads at first.

Compared with other styles of riding, this is definitely the most fast paced
and intense way of riding. 

I used to dread it, but in the past half decade or so, I have been enjoying it more and more.

There's a street that's nearby where the traffic volume is high, and I actually
enjoy riding on it. I can see how addictive this feeling could be. In fact,
there are some cyclist who become so hooked on road riding that they wish to
give everyone a hit of their particular life style "drug".

In fact, as nutty as it sounds, some of these people want to ban all other styles of riding!

Despite a few control freaks, the cycling world is full of good people who will
teach you how to ride better. I suggest meeting up with more experienced
cyclists for rides. Also, as usual, read widely (this book is just a start) and
most of all observe others. Also, remember what worked well and what didn't
work.

Finally, don't push yourself. Your naturual fear is important to keeping you
safe. People who say that you are being irrationally fearful should be ignored.
However, don't let fear overwhelm you so much that you don't ride at all. 

You'll miss out!

% Sat Sep 24 15:48:10 PDT 2011
