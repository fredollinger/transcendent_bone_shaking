\chapter{Riding Tips}

\section{What Riding Style is Best}

There are as many different riding styles as there are number of riders.

The important thing about riding style is to pick one that make you happy and comfortable.

Don't allow anyone to tell you how to ride!

Watch out for a self-professed experts who claim to have the magic key that will make you invulnearable on a bike.

Rather you should use your own reason and instincts to ride. At the same time,
you should read widely, watch other cyclists, and discuss riding strategies
with others. Overtime, with experience, you'll feel more comfortable on a
bicycle, and you'll be happier riding.

I feel like I learn something new from each cyclist that I encounter. 

\section{Salmoning}

Salmoning refers to riding against the flow of traffic. It's most commonly done
by those who ride BMX bicycles, but it is an option to all kinds of cyclists.

The main advantage of salmoning is the added feeling of safety due to the
ability of the "fish" to see the traffic as it comes and to get out of the way
if someone tries to ride you off the road.

Another added impetus toward salmoning is antiquated lessons from the
nineteen seventies which taught people to ride toward traffic.  

There are many who say that salmoning is dangerous, but I have not seen a study on this to date.

One of the main problems with salmoning, however, is that the "fish" will come
into conflict with cyclists of every other riding style. 

Also, since the salmon can see oncoming traffic, while other cyclists can not.
Thus, it's generally agreed upon that the salmon is mostly responsible for
safety and must ride further out into traffic. This double danger often negates
the percieved safety of salmoning. 

\section{Sidewalk Riding}

This is another riding style that is often derided, and seen as being highly dangerous.


In my opinion, riding on the sidewalk in either direction need not be dangerous.

The main things that cause conflicts in this   
